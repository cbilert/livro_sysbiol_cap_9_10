\chapter{Biologia de Sistemas}

A Biologia tem como finalidade entender os sistemas biológicos em detalhes para permitir predizer quantitativamente o seu desempenho, incluindo os efeitos de transformações e perturbações. Durante varias décadas, disciplinas como a Zoologia, Ecologia e Botânica, têm observado o comportamento de seres vivos e populações, já a Biologia Molecular e a Genética têm tentado explanar o funcionamento de elementos celulares individuais sobre o modo como os diversos elementos celulares interagem entre si e com o ambiente.

Nos últimos dez anos, tornou-se clara a necessidade de pesquisar, de forma metódica, o modo como processos celulares se integram para permitir a vida ao nível da célula, do tecido, do órgão, do organismo e do ecossistema. A obtenção massiva de informações em biologia molecular e celular tem levado a uma transferência da biologia centrada em moléculas para uma biologia centrada em redes, desenvolvendo-se então uma nova área chamada \textbf{Biologia de Sistemas}.

 A Biologia de Sistemas se refere ao estudo de fenômenos biológicos complexos como biologia quantitativa, redes moleculares complexas, modelos qualitativos e quantitativos preditivos, redes de regulação gênica, redes metabólica, entre outras, que buscam fornecer uma simplificação e abstração sobre eventos e princípios que formam o arcabouço necessário para a análise de redes moleculares~\textit{in vivo} que permitem à formulação de experimentos que aperfeiçoam a aquisição de dados ricos em informação, a determinação os parâmetros das redes obtidas por dados experimentais e a diferenciação, com base nos dados obtidos, de modelos plausíveis~\cite{Nicholson2006}.

Sendo assim a Biologia de Sistemas pode ser explicada como um ramo da biologia que busca descobrir e impetrar propriedades biológicas que emergem da interação entre diversos dados de um sistema.

Segundo o livro \textit{"Systems Biology in practice"}~\cite{Klipp2005} a  Biologia de Sistemas tem em vista a compreensão dos sistemas vivos considerando-os, decorrência de fluxos de massa, energia e informação, que variam no tempo, no espaço e na necessidade de determinado comportamento ou ambiente, alterando as condutas e reações em resposta ao meio em que está envolvido.

A Figura~\ref{fig:sysbiols} demonstra que o atual grau de conhecimento em sistemas biológicos é maior na base da pirâmide e seu avanço é menor à medida que se progride para o cume, devido à complexidade da informação. Desta forma a disciplina Biologia de Sistemas é essencial para se obter o grau necessário para se chegar ao topo desta pirâmide, uma vez que esta descreve os vários níveis de conhecimento, desde as sequências de nucleotídeos que são inerente aos genomas, passando pelo entendimento da função das macromoléculas biológicas como proteínas e RNAs, até à incorporação de toda a ciência que permite compreender um ecossistema e as atividades envolvidas a nível molecular~\cite{Klipp1996, Nielsen2001).

A Biologia de Sistemas surgiu com o intuito de aprimorar não só armazenamento e interpretação de dados, ela também permite a modelagem dos dados, uma vez que, a ciência matemática e a inteligência computacional permitem compreender melhor os processos envolvidos, fazendo previsões baseadas no desenvolvimento e no resultado de fatores externos derivados do comportamento de um determinado sistema; que podem envolver investigadores de segmentos muito distintos, conforme pode ser observado na Figura~\ref{fig:sysbiols1}, apesar dos desafios envolvidos no canal de comunicação entre essas áreas.

Assim como qualquer ação advinda da natureza, processos complexos como os sistemas biológicos podem explicados pela física e devem obedecer as suas leis. A termodinâmica é uma área da física que possui melhor analogia com esses sistemas e sua compreensão traz uma visão mais clara do funcionamento deles.

Para~citeonline{Willians1981} sistemas biológicos são sistemas abertos que permitem trocas de energia e matéria com o meio externo, desde o nível celular até o nível de sistemas no que se refere à evolução. A vida em sua forma mais simples demanda energia mesmo em repouso, uma vez a maquinaria bioquímica e termodinâmica envolvida estão permanentemente extraindo e utilizando energia, que têm implicações profundas na maneira como o ser vivo decompõe a energia absorvida do seu meio para mante-se organizado.

A termodinâmica é um ramo da ciência que se dedica ao estudo da energia e seus efeitos, neste sentido a vida traz desafios, mas obedece as Leis da Termodinâmica. A primeira lei da termodinâmica diz que a energia é conservada, não podendo ser criada ou destruída, apenas transformada. Já a segunda lei da termodinâmica diz que um processo será espontâneo se o caos do sistema aumentar, e isso está diretamente ligado a Entropia\footnote{} que está relacionada com a desordem de um sistema, ou seja, processos espontâneos convertem ordem em desordem ~\cite{Armostrong1978, Garrote2005, Vasconcelos2005).

A termodinâmica explica os princípios físicos e químicos reguladores do comportamento das macromoléculas e moleculares biológicos adjuntos, responsáveis por papéis fundamentais em todos os seres vivos, bem como a estabilidade de proteínas e ácidos nucleicos, velocidade e mecanismo de biorreações, transporte de moléculas por meio de membranas biológicas e a descrição da estrutura e reatividade de sistemas biológicos em geral. 

Contudo organismos vivos, não seguem todas as leis da termodinâmica, isto significa, que os sistemas vivos nem sempre estão em equilíbrio, ou seja, o metabolismo celular conecta os processos espontâneos de Catabolismo (conjunto de ações metabólicas que quebram grandes moléculas) com os processos não espontâneos de Anabolismo (processo pelo qual o corpo aproveita a energia liberada pelo catabolismo para sintetizar moléculas complexas). O catabolismo libera a energia e o anabolismo demanda a energia. Em termos termodinâmicos, o metabolismo mantém o balanço, já que os processos metabólicos são reações químicas que envolvem frequentemente a geração de calor~\cite{Dickerson1969, Demetrius2000). 

Mais do que entender cada sistema e as leis que conjecturam e regulam as relações entre biomoléculas e seus metabolismos, à ambição da disciplina de Biologia de Sistemas é compreender e ser capaz de predizer o funcionamento de qualquer sistema, célula, tecido ou organismo. Seu enfoque está relacionado ao estudo da influência mútua entre os dados de fisiologia, genômica, proteômica e metabolômica, admitindo, assim, a construção de um modelo que conceba e simule a fisiologia de todo o sistema e que elucide a forma e a função dos organismos vivos~\citep{Gutierrez2005). 

Dessa forma a finalidade básica dessa abordagem é impetrar uma visão holística de um organismo vivo, de tal maneira que as respostas dos elementos desse sistema a qualquer tipo de pertubação possa ser exatamente predita~\citep{Rossignol2006}.


\begin{figure}
\begin{center}
\includegraphics[width=0.6\textwidth]{figuras/piramide_de_complexidade.pdf}
\end{center}
\caption{Pirâmide de conhecimento dos sistemas biológicos.\footnote{Fonte. Página e-Escola: http://e-escola.ist.utl.pt/}}
\label{fig:sysbiol}
\end{figure}

\begin{figure}
\begin{center}
\includegraphics[width=0.6\textwidth]{Interdisciplinaridade_Biologia_Sistema.pdf}
\end{center}
\caption{Interdisciplinaridade de Biologia de Sistema. \footnote{Fonte. Página e-Escola: http://e-escola.ist.utl.pt/}}
\label{fig:sysbiol1}
\end{figure}


