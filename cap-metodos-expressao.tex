\chapter{METODOS PARA MEDIR A EXPRESSÃO DOS GENES}
Expressão gênica é o processo onde a sequência de DNA é traduzida em outras estruturas tais como  proteínas e RNA. A tradução dessa sequência em proteína ou RNA é chamado transcrição.
Durante o processo de transcrição, as sequência de nucleotídeos presentes nos genes são utilizados para codificar uma molécula de RNA mensageiro (mRNA ou RNAm) e esta, por sua vez, realiza a síntese de proteínas na região citoplasmática da célula, especificando quais aminoácidos que irão compor as proteínas individuais.(ver capitulo sobre Dogma Central da Biologia Molecular).

A determinação da concentração de mRNA presente na célula permite obter o nível de expressão do gene analisado. A comparação dos níveis de mRNA de um gene numa célula permite associar esse gene a determinados eventos que ocorrem durante o processo de desenvolvimento celular ou alterações fisiológicas durante as fases de desenvolvimento de seres vivos.~\cite{Marcus2008}

Os níveis de mRNA podem ser medidos por métodos de quantização como Northern Blot (~\cite{Speed}), PCR quantitativo() e RT-PCR (PCR quantitativo em tempo-real)~\cite{Heid1996), também são utilizdos os métodos de microarrays~\cite{Slonim2009}, SAGE (Serial Analisys, of Gene Expression)~\cite{Velculescu1995}, RNAseq~\cite{Dobin2013}, ~\cite{Krupp2012} entre outros.

\section{Northern Blot}

\section{PCR quantitativo e PCR quantitativo em Tempo-Real}

\section{Microarrays}

\section{SAGE - Serial Analisys of Gene Expression}

\section{RNAseq}
